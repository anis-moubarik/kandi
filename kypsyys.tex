% --- Template for thesis / report with tktltiki2 class ---
% 
% last updated 2013/02/15 for tkltiki2 v1.02

\documentclass[finnish]{tktltiki2}

% tktltiki2 automatically loads babel, so you can simply
% give the language parameter (e.g. finnish, swedish, english, british) as
% a parameter for the class: \documentclass[finnish]{tktltiki2}.
% The information on title and abstract is generated automatically depending on
% the language, see below if you need to change any of these manually.
% 
% Class options:
% - grading                 -- Print labels for grading information on the front page.
% - disablelastpagecounter  -- Disables the automatic generation of page number information
%                              in the abstract. See also \numberofpagesinformation{} command below.
%
% The class also respects the following options of article class:
%   10pt, 11pt, 12pt, final, draft, oneside, twoside,
%   openright, openany, onecolumn, twocolumn, leqno, fleqn
%
% The default font size is 11pt. The paper size used is A4, other sizes are not supported.
%
% rubber: module pdftex

% --- General packages ---

\usepackage[utf8]{inputenc}
\usepackage[T1]{fontenc}
\usepackage{lmodern}
\usepackage{microtype}
\usepackage{amsfonts,amsmath,amssymb,amsthm,booktabs,color,enumitem,graphicx}
\usepackage[pdftex,hidelinks]{hyperref}
 \usepackage{algpseudocode}

% Automatically set the PDF metadata fields
\makeatletter
\AtBeginDocument{\hypersetup{pdftitle = {\@title}, pdfauthor = {\@author}}}
\makeatother

% --- Language-related settings ---
%
% these should be modified according to your language

% babelbib for non-english bibliography using bibtex
\usepackage[fixlanguage]{babelbib}
\selectbiblanguage{finnish}

% add bibliography to the table of contents
\usepackage[nottoc]{tocbibind}
% tocbibind renames the bibliography, use the following to change it back
\settocbibname{Lähteet}

% --- Theorem environment definitions ---

\newtheorem{lau}{Lause}
\newtheorem{lem}[lau]{Lemma}
\newtheorem{kor}[lau]{Korollaari}

\theoremstyle{definition}
\newtheorem{maar}[lau]{Määritelmä}
\newtheorem{ong}{Ongelma}
\newtheorem{alg}[lau]{Algoritmi}
\newtheorem{esim}[lau]{Esimerkki}

\theoremstyle{remark}
\newtheorem*{huom}{Huomautus}


% --- tktltiki2 options ---
%
% The following commands define the information used to generate title and
% abstract pages. The following entries should be always specified:

\title{Vakaa avioliitto -ongelma}
\author{Anis Moubarik}
\date{\today}
\level{Kypsyysnäyte}
\abstract{}

% The following can be used to specify keywords and classification of the paper:

\keywords{pariutus, gale-shapley, avioliitto-peli}
\classification{} % classification according to ACM Computing Classification System (http://www.acm.org/about/class/)
                  % This is probably mostly relevant for computer scientists

% If the automatic page number counting is not working as desired in your case,
% uncomment the following to manually set the number of pages displayed in the abstract page:
%
% \numberofpagesinformation{16 sivua + 10 sivua liitteissä}
%
% If you are not a computer scientist, you will want to uncomment the following by hand and specify
% your department, faculty and subject by hand:
%
% \faculty{Matemaattis-luonnontieteellinen}
% \department{Tietojenkäsittelytieteen laitos}
% \subject{Tietojenkäsittelytiede}
%
% If you are not from the University of Helsinki, then you will most likely want to set these also:
%
% \university{Helsingin Yliopisto}
% \universitylong{HELSINGIN YLIOPISTO --- HELSINGFORS UNIVERSITET --- UNIVERSITY OF HELSINKI} % displayed on the top of the abstract page
% \city{Helsinki}
%


\begin{document}

% --- Front matter ---

\maketitle        % title page
\newpage          % clear page after the table of contents


% --- Main matter ---

\section{Johdanto}
Vakaa avioliitto -ongelma sivuaa kombinatoriikkaa ja peliteoriaa. Ongelma on seuraava: Olkoon olemassa $n$ miestä ja $n$ naista ja miehet ovat pistäneet naiset järjestykseen mieluisimmasta parista huonoimpaan ja jokainen nainen on pistänyt miehet vastaavaan järjestykseen, tätä tilannetta kutsutaan \emph{avioliittopeliksi}. \emph{Pari} on tässä kontekstissa siis mies-nais pari, tätä kutsutaan myös \emph{avioliitoksi}. Tehtävänä on pariuttaa miehet ja naiset niin, että meillä ei ole kahta vastakkaisen sukupuolen henkilöä, jotka olisivat mielummin pari keskenään kuin pysyisivät avioliitossaan nykyisessä pariutuksessa. Avioliitot ovat vakaita, kun näitä \emph{estepareja} ei esiinny. Pariutus on joukko pareja ja parit voidaan pariuttaa Gale--Shapley-algoritmilla, joka esitellään myöhemmin. Vakaassa pariutuksessa on siis tarkoitus pariuttaa kahden joukon alkiot niin että jokin kriteeri täyttyy, alkuperäisessä vakaa avioliitto -ongelmassa tuo kriteeri on parien vakaus suhteessa alkioiden mieltymyksiin.

Ongelmalla on monia käytännön sovelluksia. Se voidaan yleistää kaikkiin \emph{kaksipuolisiin markkinoihin}, joissa eettisistä tai laillisista syistä ei käytetä vaihtovälineenä rahaa. Kaksipuolisilla markkinoilla tarkoitetaan tilannetta, jossa on olemassa alusta joka palvelee kahta erillistä ryhmää, jotka toimittavat toisilleen hyödykkeitä. Seuraavat markkinat ovat hyviä esimerkkejä: elinluovuttajat ja vastaanottajat, työntekijät ja työnantajat sekä opiskelijat ja korkeakoulut. Elinluovutusten tapauksessa haaste on se, että luovuttaja-potilas kaksikot eivät ole yhteensopivia, voimme kuitenkin käyttää vakaa avioliitto -ongelman ratkaisuja elinluovutuksissa. Jos on olemassa kaksikot $A$, $B$ ja $C$, eli jokainen kaksikoista sisältää luovuttajan ja potilaan, mutta syystä tai toisesta elin ei sovi potilaalle, voidaan soveltaa ratkaisuja tähän tapaukseen ja saada vakaita pariutuksia potilaitten ja luovuttajien välille. Oletetaan, että $A$:n luovuttaja sopii $B$:lle, $B$:n $C$:lle ja $C$:n $A$:lle, nyt saamme ketjun ja vakaan pariutuksen, jossa jokainen potilas saa uuden elimen. Näitä \emph{kaksipuolisia pariutusmarkkinoita} tutkimme paperissa.

Vastaan tulee myös pieniä variaatioita klassiseen avioliitto-ongelmaan, kuten erisuuruiset joukot ja alkioiden hyvin erilaiset mieltymykset, eli tapaukset joissa he eivät halua pariutua tietyn henkilön kanssa ollenkaan. Näissä tapauksissa vakauden määritelmää muokataan niin että pariutukset joissa alkio on ilman paria voi olla myös vakaa. Erisuuruisten joukkojen tapauksessa suuremman joukon alkioista jää pariuttamatta suuremman ja pienemmän joukon alkioiden lukumärään erotuksen verran alkioita.

Gale ja Shapley esittivät vuonna 1962 ilmestyneessä artikkelissa \cite{gale62a} algoritmin alkuperäiselle ongelmalle, joka löytää yhden vakaan pariutuksen jokaiselle olemassa olevalle avioliittopelille. Jokaisella avioliittopelillä on siis vähintään yksi vakaa pariutus. Algoritmi toimii oletuksena "miehet kosii, naiset hylkää" periaatteella. Tämä antaa kosijalle, tässä tapauksessa siis miehille, aina parhaan mahdollisen parin mitä he voivat pelin kontekstissa saada. Naisten kosiessa roolit tietysti vaihtuvat ja pariutukset ovat silloin naisille optimoituja.
Peliteoria tulee esille, kun joukkojen alkiot vääristelevät mieltymyksiään. Tähän vääristelyyn Gale--Shapley-algoritmi ei osaa puuttua ja se nostaakin kysymyksiä kuinka siihen pitäisi reagoida ja miten ylipäätään alkiot vääristelystä hyötyvät, jos hyötyvät ollenkaan.


% --- Back matter ---
%
% bibtex is used to generate the bibliography. The babplain style
% will generate numeric references (e.g. [1]) appropriate for theoretical
% computer science. If you need alphanumeric references (e.g [Tur90]), use
%
% \bibliographystyle{babalpha-lf}
%
% instead.

\bibliographystyle{babplain-lf}
\bibliography{ref}


\end{document}