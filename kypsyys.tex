% --- Template for thesis / report with tktltiki2 class ---
% 
% last updated 2013/02/15 for tkltiki2 v1.02

\documentclass[finnish]{tktltiki2}

% tktltiki2 automatically loads babel, so you can simply
% give the language parameter (e.g. finnish, swedish, english, british) as
% a parameter for the class: \documentclass[finnish]{tktltiki2}.
% The information on title and abstract is generated automatically depending on
% the language, see below if you need to change any of these manually.
% 
% Class options:
% - grading                 -- Print labels for grading information on the front page.
% - disablelastpagecounter  -- Disables the automatic generation of page number information
%                              in the abstract. See also \numberofpagesinformation{} command below.
%
% The class also respects the following options of article class:
%   10pt, 11pt, 12pt, final, draft, oneside, twoside,
%   openright, openany, onecolumn, twocolumn, leqno, fleqn
%
% The default font size is 11pt. The paper size used is A4, other sizes are not supported.
%
% rubber: module pdftex

% --- General packages ---

\usepackage[utf8]{inputenc}
\usepackage[T1]{fontenc}
\usepackage{lmodern}
\usepackage{microtype}
\usepackage{amsfonts,amsmath,amssymb,amsthm,booktabs,color,enumitem,graphicx}
\usepackage[pdftex,hidelinks]{hyperref}
 \usepackage{algpseudocode}

% Automatically set the PDF metadata fields
\makeatletter
\AtBeginDocument{\hypersetup{pdftitle = {\@title}, pdfauthor = {\@author}}}
\makeatother

% --- Language-related settings ---
%
% these should be modified according to your language

% babelbib for non-english bibliography using bibtex
\usepackage[fixlanguage]{babelbib}
\selectbiblanguage{finnish}

% add bibliography to the table of contents
\usepackage[nottoc]{tocbibind}
% tocbibind renames the bibliography, use the following to change it back
\settocbibname{Lähteet}

% --- Theorem environment definitions ---

\newtheorem{lau}{Lause}
\newtheorem{lem}[lau]{Lemma}
\newtheorem{kor}[lau]{Korollaari}

\theoremstyle{definition}
\newtheorem{maar}[lau]{Määritelmä}
\newtheorem{ong}{Ongelma}
\newtheorem{alg}[lau]{Algoritmi}
\newtheorem{esim}[lau]{Esimerkki}

\theoremstyle{remark}
\newtheorem*{huom}{Huomautus}


% --- tktltiki2 options ---
%
% The following commands define the information used to generate title and
% abstract pages. The following entries should be always specified:

\title{Vakaa avioliitto -ongelma}
\author{Anis Moubarik}
\date{\today}
\level{Kypsyysnäyte}
\abstract{}

% The following can be used to specify keywords and classification of the paper:

\keywords{pariutus, gale-shapley, avioliitto-peli}
\classification{} % classification according to ACM Computing Classification System (http://www.acm.org/about/class/)
                  % This is probably mostly relevant for computer scientists

% If the automatic page number counting is not working as desired in your case,
% uncomment the following to manually set the number of pages displayed in the abstract page:
%
% \numberofpagesinformation{16 sivua + 10 sivua liitteissä}
%
% If you are not a computer scientist, you will want to uncomment the following by hand and specify
% your department, faculty and subject by hand:
%
% \faculty{Matemaattis-luonnontieteellinen}
% \department{Tietojenkäsittelytieteen laitos}
% \subject{Tietojenkäsittelytiede}
%
% If you are not from the University of Helsinki, then you will most likely want to set these also:
%
% \university{Helsingin Yliopisto}
% \universitylong{HELSINGIN YLIOPISTO --- HELSINGFORS UNIVERSITET --- UNIVERSITY OF HELSINKI} % displayed on the top of the abstract page
% \city{Helsinki}
%


\begin{document}

% --- Front matter ---

\maketitle        % title page
\newpage          % clear page after the table of contents


% --- Main matter ---

\section{Johdanto}
Vakaa avioliitto -ongelma on Galen ja Shapleyn vuonna 1962 tulleessa artikkelissa \cite{gale62a}, esille tuoma ongelma joka sivuuttaa kombinatoriikkaa ja peliteoriaa. Ongelma on seuraava. Olkoon meillä $n$ miestä ja $n$ naista, miehet ovat pistäneet naiset järjestykseen mieluisimmasta morsiammesta huonoimpaan ja vastaavasti jokainen nainen on pistänyt miehet järjestykseen mieluisimmasta sulhasesta huonoimpaan. Tehtävänä on pariuttaa miehet ja naiset niin, että meillä ei ole estepareja, ja kun näitä ei esiinny ovat avioliitot vakaita.\\
Ongelmalla on monia käytännön sovelluksia, kuten elinluovutukset, joissa osapuolina ovat miesten ja naisten sijaan luovuttajat ja vastaanottajat sekä omien mieltymysten sijaan vastaanottajien kriteerit ovat elinten sopivuus heille. 

Gale ja Shapley esittivät algoritmin, joka toteuttaa yhden vakaan pariutuksen jokaiselle olemassa olevalle avioliittopelille, tämä tarkoittaa sitä että jokainen avioliittopeli sisältää vakaan pariutuksen. Algoritmi toimii oletuksena miehet kosii; naiset hylkää periaatteella, tämä antaa kosijalle, tässä tapauksessa siis miehille, aina parhaan mahdollisen parin mitä he voivat pelin kontekstissa saada.
Peliteoria nousee esille ongelmassa, koska joukkojen alkiot voivat vääristellä mieltymyksiään, tästä nousee kysymyksiä kuinka tähän olisi reagoitava, algoritmi ei pysty siihen, ja kuinka alkiot vääristelystä hyötyvät.

Tekstin tarkoituksena on esitellä vakaa avioliitto -ongelma, sen ratkaisut ja sovellukset. Notaatio on matemaattisesti kevyttä ja ongelma on, sovellusten kautta, helposti lähestyttävissä.

\section{Vakaat pariutukset}
Olkooon $M$ \emph{miesten} joukko ja $N$ naisten joukko, missä $|M| = |N| = n$. Avioliitosta tai parista puhuttaessa tarkoitamme paria $(m, n) \in M \times N$. Avioliittopelissä jokaisella joukkojen alkiolla on mieltymykset toisen joukon alkioista parhaimmasta kumppanista huonoinmpaan. Olkoon meillä alkiot $\{x, y, z\} \subset N$ mieltymykset $m$:lle voidaan kuvata seuraavasti; $x >_m y >_m z$ mieltymykset ovat myös transitiivisia eli $x >_m z$ pätee. Merkitään pariutusten joukkoa $\mu$:llä, eli $(m,n) \in \mu$. Pariutus $\mu$ on joukko $\mu \subseteq M \times N$, missä pariutus on injektio $\mu : M \mapsto N$

\begin{maar}
\emph{Vakaa} pariutus on ekvivalentti \emph{esteparin} poissaololle. Esteparin määritelmä; olkoon meillä neljä alkiota avioliittopelistä $\{n_1, n_2\} \subset N, \{m_1, m_2\} \subset M$ ja pariutus $\{(n_1, m_1), (n_2, m_2)\} \subset \mu$. Jos $m_2 >_{n_1} m_1$ ja $n_1 >_{m_2} n_2$,niin $(n_1, m_2)$ on estepari. Haluamme välttää epävakautta, jotta pariutukset olisivat mahdollisimman kestäviä. Ongelma on löytää pariutus $\delta$ joukoille $N$ ja $M$, joka olisi vakaa.
\end{maar}
Pariutuksilla ja pareilla on erillaisia ominaisuuksia. Paras-miehelle pari on pari, jossa mies saa haluamansa naisen ja vastaavasti paras-naiselle pari on pari, jossa nainen saa haluamansa miehen. Pariutuksissa voi esiintyä jommankumman joukon ylivaltaa, jos $\forall j \in N$, $j$ saavat haluamansa parin, tätä kutsutaan $N$-joukon ylivallaksi ja samoin $\forall i \in M$, $i$ saavat haluamansa parin, tällöin se on $M$-joukon ylivalta. Joskus ylivaltoja kutsutaan myös \emph{Nais}- tai \emph{Miesylivallaksi}. Jos ylivaltaa ei synny pariutuksessa, voidaan pariutusta kutsua ylivaltavapaaksi.
Päinvastoin jos mies tai nainen saa huonoimman tai vähiten haluamansa parin, kutsutaan sitä \emph{huonoin-naiselle} tai -\emph{miehelle} pariksi. Samoin pariutusta kutsutaan huonoin-naiselle tai -miehelle pariksi jos kaikki jommankumman joukon alkioista saavat vähiten haluamansa parit.

Esittelemme myöhemmin teoreeman, jonka mukaan jokaisesta avioliittopelistä löytyy ainakin yksi vakaa pariutus. Ensin kuitenkin esittelemme algoritmin joka tarkastaa pariutuksen vakauden.
Olkoon $x = |M|$, $m \in M$ ja $n \in N$. Olkoon $\mu$ meidän pariutus ja jos $n$ ja $m$ on pariutettu $\mu$:ssa, niin merkitään sitä $n = p_\mu(m)$.
\begin{algorithmic}
	\For{$i = 1$ to $x$}
		\ForAll {$n$ niin että $m$ suosii $n$:ää ennemmin kuin $p_\mu(m)$}
			\If{$n$ suosii $m$:ää ennemmin kuin $p_\mu(n)$}
				\State \Return pariutus on epävakaa
			\EndIf
		\EndFor
	\EndFor\\
	\Return pariutus on vakaa
	\cite[p. 8]{gusfield1989stable}.
\end{algorithmic}

Etsimme siis tapauksia jossa mies suosii naista enemmän kuin omaa pariaan pariutuksessa $p_\mu$.
Jos löydämme tälläisen miehen, tarkastelemme jos nainen jota hän suosii, suosii myös miestä enemmän kuin pariaan, jos tämä pitää paikkansa, ei pariutus voi olla vakaa.
Huonoimmassa tapauksessa käymme läpi kaikki miehet ja kaikki naiset paitsi yhden, eli $x\cdot(x-1)$, tästä saamme aikavaativuudeksi $O(x^2)$.


% --- Back matter ---
%
% bibtex is used to generate the bibliography. The babplain style
% will generate numeric references (e.g. [1]) appropriate for theoretical
% computer science. If you need alphanumeric references (e.g [Tur90]), use
%
% \bibliographystyle{babalpha-lf}
%
% instead.

\bibliographystyle{babplain-lf}
\bibliography{ref}


\end{document}