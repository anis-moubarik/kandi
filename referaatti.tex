% --- Template for thesis / report with tktltiki2 class ---

\documentclass[finnish]{tktltiki2}

% tktltiki2 automatically loads babel, so you can simply
% give the language parameter (e.g. finnish, swedish, english, british) as
% a parameter for the class: \documentclass[finnish]{tktltiki2}.
% The information on title and abstract is generated automatically depending on
% the language, see below if you need to change any of these manually.
% 
% Class options:
% - grading                 -- Print labels for grading information on the front page.
% - disablelastpagecounter  -- Disables the automatic generation of page number information
%                              in the abstract. See also \numberofpagesinformation{} command below.
%
% The class also respects the following options of article class:
%   10pt, 11pt, 12pt, final, draft, oneside, twoside,
%   openright, openany, onecolumn, twocolumn, leqno, fleqn
%
% The default font size is 11pt. The paper size used is A4, other sizes are not supported.
%
% rubber: module pdftex

% --- General packages ---

\usepackage[utf8]{inputenc}
\usepackage{lmodern}
\usepackage{microtype}
\usepackage{amsfonts,amsmath,amssymb,amsthm,booktabs,color,enumitem,graphicx}
\usepackage[pdftex,hidelinks]{hyperref}

% Automatically set the PDF metadata fields
\makeatletter
\AtBeginDocument{\hypersetup{pdftitle = {\@title}, pdfauthor = {\@author}}}
\makeatother

% --- Language-related settings ---
%
% these should be modified according to your language

% babelbib for non-english bibliography using bibtex
\usepackage[fixlanguage]{babelbib}
\selectbiblanguage{finnish}

% add bibliography to the table of contents
\usepackage[nottoc,numbib]{tocbibind}
% tocbibind renames the bibliography, use the following to change it back
\settocbibname{Lähteet}

% --- Theorem environment definitions ---

\newtheorem{lau}{Lause}
\newtheorem{lem}[lau]{Lemma}
\newtheorem{kor}[lau]{Korollaari}

\theoremstyle{definition}
\newtheorem{maar}[lau]{Määritelmä}
\newtheorem{ong}{Ongelma}
\newtheorem{alg}[lau]{Algoritmi}
\newtheorem{esim}[lau]{Esimerkki}

\theoremstyle{remark}
\newtheorem*{huom}{Huomautus}


% --- tktltiki2 options ---
%
% The following commands define the information used to generate title and
% abstract pages. The following entries should be always specified:

\title{Vakaa avioliitto -ongelma}
\author{Anis Moubarik}
\date{\today}
\level{Referaatti}
\abstract{}

% The following can be used to specify keywords and classification of the paper:

\keywords{vakaa avioliitto -ongelma, vakaat parit}
\classification{} % classification according to ACM Computing Classification System (http://www.acm.org/about/class/)
                  % This is probably mostly relevant for computer scientists

% If the automatic page number counting is not working as desired in your case,
% uncomment the following to manually set the number of pages displayed in the abstract page:
%
% \numberofpagesinformation{16 sivua + 10 sivua liitteissä}
%
% If you are not a computer scientist, you will want to uncomment the following by hand and specify
% your department, faculty and subject by hand:
%
% \faculty{Matemaattis-luonnontieteellinen}
% Outo ääkkösongelma, jos näitä ei määrittele tässä.
 \department{Tietojenkäsittelytieteen laitos} 
 \subject{Tietojenkäsittelytiede}
%
% If you are not from the University of Helsinki, then you will most likely want to set these also:
%
% \university{Helsingin Yliopisto}
% \universitylong{HELSINGIN YLIOPISTO --- HELSINGFORS UNIVERSITET --- UNIVERSITY OF HELSINKI} % displayed on the top of the abstract page
% \city{Helsinki}
%

\begin{document}

% --- Front matter ---

\maketitle        % title page
\makeabstract     % abstract page

\newpage          % clear page after the table of contents


% --- Main matter ---

\section{Mikä on vakaa avioliitto -ongelma}

Teksti pohjautuu Michel Balinskin ja Guillaume Ratierin, Of Stable Marriages and Graphs, And Strategy and Polytopes, artikkeliin.\\
Kun puhumme avioliitosta tässä kontekstissa, tarkoitamme kahden alkion pariutumista kahdesta erillisestä joukosta. $m \in M$ \emph{"miehet"} ja $n \in N$ \emph{"naiset"}, pariutus on $(m, n)$.
Kutsutaan peliä avioliitto peliksi, kun kummankin joukon alkiot sijottavat vastapuolisen joukon alkion mieltymys järjestykseen $1 \ldots n$, missä $n$ on vastapuolisen joukon alkioiden määrä. Pelin tarkoitus on löytää pari jokaiselle alkiolle. Vakaa avioliitto -ongelman kysymys on, voiko mille tahansa alkioiden mieltymyksille löytää vakaat pariutukset? Pariutus $(m_{1}, w_{1})$ on vakaa, joss ei ole olemassa alkiota $w_{2}$ tai $m_{2}$, jossa $m_{1}$ ja $w_{2}$ pariutuisivat mieluummin keskenään, kuin omien pariensa kanssa, ja $m_{2}$ ja $w_{1}$ pariutuisivat mieluummin keskenään, kuin omien pariensa kanssa, tälläisiä pareja kutsutaan estepareiksi. Haluamme siis välttää epävakautta, jotta pariutukset olisivat mahdollisimman kestäviä.

Tarkoituksena on lähestyä ongelmaa suunnattujen verkkojen kautta. Verkot voidaan ajatella matriiseina, joissa riveinä ovat miehet ja sarakkeina naiset ja nuoli osoittaa solusta soluun, huonoimmasta parista parhaimpaan.


\section{Vakaa pari}
Jos haluamme esittää ongelman suunnattuna verkkona, meillä on kaksi ääreellistä joukkoa, 
$M = \{m_{1}, m_{2},..., m_{|M|}\}$, ja $N = \{n_{1}, n_{2},...,n_{|N|}\}$. Jokaisella joukon jäsenellä on selvät mieltymykset toisen joukon jäsenistä, ja sijoitus parhaasta parista huonoimpaan. Joukkojen jäsenten on mahdollista jäädä selibaateiksi, mutta jos tälläistä jäsentä ei ole ollenkaan vakaus on ekvivalentti esteparin $(m, w)$ poissaololle.\\
Paras-miehelle solmu on pari, jossa mies saa haluamansa naisen, ja paras-naiselle solmu, jossa nainen saa haluamansa miehen. Paras-miehelle tai -naiselle ylivalta tapahtuu, kun jokainen mies pariutuksessa $\mu$ saa haluamansa naisen, tai päinvastoin, jokainen nainen saa pariutuksessa $\mu$ haluamansa miehen. Jos tälläistä ylivaltaa ei tapahdu, pariutusta voidaan kutsua ylivalta vapaaksi.\\
Voidaan puhua myös huonoin-miehelle tai -naiselle pariutuksesta, jossa mies tai nainen saa vähiten haluamansa parin.


\section{Vakaitten parien löytäminen}
Esitämme alkuperäisen Gale-Shapley \emph{"miehet kosii, naiset määrää"} algoritmin.
Miehet aloittavat kosimalla suosikki naistaan. Jokainen nainen joka saa enemmän kuin yhden kosinnan hylkää kosineista miehistä kaikki paitsi suosikkinsa. Nainen ei kuitenkaan hyväksy miestä vielä, vaan pitää häntä odottavassa tilassa salliakseen mahdollisuuden sille, että parempi mies tulisi vielä kosimaan seuraavilla kierroksilla.

Seuraavassa vaiheessa miehet jotka hylättiin kosivat seuraavia vaihtoehtoja, ja naiset taas hylkäävät kaikki paitsi parhaan vaihtoehdon.

Koska niin kauan kun joku naisista ei ole saanut kosintaa tulee hylkäyksiä ja uusia kosintoja, ja mies ei voi kuin kerran kosia samaa naista, niin lopuksi jokaista naista on kosittu. Kun viimeinen nainen on saanut kosinnan, on kosimisvaihe päättynyt ja jokaisen naisen on hyväksyttävä langan päässä oleva mies. \cite[p. 12-13]{gale62a} On myös ilmeistä, että algoritmi päättyy.

Haluamme todistaa, että aiemman pelin avioliitot ovat vakaita.
Oletetaan, että Matti ja Mari eivät ole naimisissa keskenään, mutta Matti suosii Mariaa hänen oman vaimonsa yli. Matin on siis jossain vaiheessa peliä pitänyt kosia Maria ja Marin jossain vaiheessa hylätä Matti jonkun toisen, paremman parin, edestä. On siis selvää, että Mari suosii omaa aviomiestään Matin sijaan eikä epävakautta synny. \cite[p. 13]{gale62a} Algoritmin aikavaativuus on $O(n^2)$ \cite[p. 588]{Balinski}, ja vakaitten pariutusten määrä on eksponentiaalinen pelissä. \cite[p. 591]{Balinski} Täytyy myös huomauttaa, että algoritmi tuottaa tälläisenaan miehille optimaalisia tuloksia, eli tällä algoritmilla tuotettu pariutus on kyseisille miehille paras mahdollinen mitä he siinä kyseisessä avioliitto pelissä tulevat saamaan.\\
Jos halutaan tuottaa naisille optimaalisia tuloksia voidaan osat vaihtaa päittäin algoritmissa ja aloittaa alusta.


\section{Sovellukset}
Vakaa avioliitto algoritmin sovellukset ovat laajat. Sitä voidaan käyttää kaksipuolisissa markkinoissa, kuten vuokra-asunnon hakemisessa, jossa vuokraajat ja hakijat ovat pelaavat joukot, vuokraajat hakevat vuokraaja-optimaalisinta pariutusta. Vuokraajat lähettävät "pelin" alussa "kosinnan" suosikki vuokralaisehdokkaalleen, peli toimii täsmälleen samalla tavalla kuin naisten ja miesten kanssa.
Vuokralaisehdokas voi myös pelata peliä, jossa saadaan tulokseksi vuokralais-optimaalinen tulos, kun hän hakee asuntoa ja aloittaa suosikki kohteestaan.

Toinen esimerkki on elinluvotus tapaukset, jossa suurin tekijä on elimen sopivuus vastaanottajalle. Tässä tapauksessa on siis kaksi joukkoa, luovuttajat ja vastaanottajat, ja haemme vain vastaanottaja-optimaalista pariutusta.

Yleisesti ongelmaa ja sen ratkaisua voidaan soveltaa kaksipuolisissa markkinoissa, nämä ovat taloudellisia alustoja joilla on kaksi erillistä käyttäjäryhmää, jotka tarjoavat toisilleen tavaroita tai palveluita.


% --- Back matter ---
%
% bibtex is used to generate the bibliography. The babplain style
% will generate numeric references (e.g. [1]) appropriate for theoretical
% computer science. If you need alphanumeric references (e.g [Tur90]), use
%
% \bibliographystyle{babalpha}
%
% instead.

\bibliographystyle{babplain}
\bibliography{ref}

\lastpage

\end{document}