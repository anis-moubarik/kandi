% --- Template for thesis / report with tktltiki2 class ---

\documentclass[finnish]{tktltiki2}

% tktltiki2 automatically loads babel, so you can simply
% give the language parameter (e.g. finnish, swedish, english, british) as
% a parameter for the class: \documentclass[finnish]{tktltiki2}.
% The information on title and abstract is generated automatically depending on
% the language, see below if you need to change any of these manually.
% 
% Class options:
% - grading                 -- Print labels for grading information on the front page.
% - disablelastpagecounter  -- Disables the automatic generation of page number information
%                              in the abstract. See also \numberofpagesinformation{} command below.
%
% The class also respects the following options of article class:
%   10pt, 11pt, 12pt, final, draft, oneside, twoside,
%   openright, openany, onecolumn, twocolumn, leqno, fleqn
%
% The default font size is 11pt. The paper size used is A4, other sizes are not supported.
%
% rubber: module pdftex

% --- General packages ---

\usepackage[utf8]{inputenc}
\usepackage{lmodern}
\usepackage{microtype}
\usepackage{amsfonts,amsmath,amssymb,amsthm,booktabs,color,enumitem,graphicx}
\usepackage[pdftex,hidelinks]{hyperref}

% Automatically set the PDF metadata fields
\makeatletter
\AtBeginDocument{\hypersetup{pdftitle = {\@title}, pdfauthor = {\@author}}}
\makeatother

% --- Language-related settings ---
%
% these should be modified according to your language

% babelbib for non-english bibliography using bibtex
\usepackage[fixlanguage]{babelbib}
\selectbiblanguage{finnish}

% add bibliography to the table of contents
\usepackage[nottoc,numbib]{tocbibind}
% tocbibind renames the bibliography, use the following to change it back
\settocbibname{Lähteet}

% --- Theorem environment definitions ---

\newtheorem{lau}{Lause}
\newtheorem{lem}[lau]{Lemma}
\newtheorem{kor}[lau]{Korollaari}

\theoremstyle{definition}
\newtheorem{maar}[lau]{Määritelmä}
\newtheorem{ong}{Ongelma}
\newtheorem{alg}[lau]{Algoritmi}
\newtheorem{esim}[lau]{Esimerkki}

\theoremstyle{remark}
\newtheorem*{huom}{Huomautus}


% --- tktltiki2 options ---
%
% The following commands define the information used to generate title and
% abstract pages. The following entries should be always specified:

\title{Vakaa avioliitto ongelma}
\author{Anis Moubarik}
\date{\today}
\level{Referaatti}
\abstract{}

% The following can be used to specify keywords and classification of the paper:

\keywords{vakaa avioliitto ongelma, vakaat parit}
\classification{} % classification according to ACM Computing Classification System (http://www.acm.org/about/class/)
                  % This is probably mostly relevant for computer scientists

% If the automatic page number counting is not working as desired in your case,
% uncomment the following to manually set the number of pages displayed in the abstract page:
%
% \numberofpagesinformation{16 sivua + 10 sivua liitteissä}
%
% If you are not a computer scientist, you will want to uncomment the following by hand and specify
% your department, faculty and subject by hand:
%
% \faculty{Matemaattis-luonnontieteellinen}
% Outo ääkkösongelma, jos näitä ei määrittele tässä.
 \department{Tietojenkäsittelytieteen laitos} 
 \subject{Tietojenkäsittelytiede}
%
% If you are not from the University of Helsinki, then you will most likely want to set these also:
%
% \university{Helsingin Yliopisto}
% \universitylong{HELSINGIN YLIOPISTO --- HELSINGFORS UNIVERSITET --- UNIVERSITY OF HELSINKI} % displayed on the top of the abstract page
% \city{Helsinki}
%

\begin{document}

% --- Front matter ---

\maketitle        % title page
\makeabstract     % abstract page

\newpage          % clear page after the table of contents


% --- Main matter ---

\section{Mikä on vakaa avioliitto ongelma}

% Write some science here.

Avioliitto ongelmassa on kyse kahdesta erillisestä joukosta ja näiden joukkojen keskinäisestä parituksesta, ongelmana on löytää paritus joka on vakaa. Paritus on vakaa, kun ei löydy paria $(m, w)$ jossa $m$ ja $w$ olisivat erikseen paremmassa asemassa kuin yhdessä.

Tarkoituksena on lähestyä ongelmaa suunnattujen verkkojen kautta. Verkot voidaan ajatella matriiseina, joissa riveinä ovat miehet ja sarakkeina naiset ja nuoli osoittaa solusta soluun, huonoimmasta parista parhaimpaan.


\section{Vakaat parit}
Jos haluamme esittää ongelman suunnattuna verkkona, meillä on kaksi ääreellistä joukkoa, 
$M = \{m_{1}, m_{2},..., m_{|M|}\}$, ja $N = \{n_{1}, n_{2},...,n_{|N|}\}$. Jokaisella joukon jäsenellä on selvät mieltymykset toisen joukon jäsenistä, ja sijoitus parhaasta parista huonoimpaan. Paritus verkossa $\Gamma$ on parit $(m, w)$ $m \in M$, $n \in N$, niin että $w$ on $m$:lle sopiva pari ja päinvastoin. Joukkojen jäsenten on mahdollista jäädä selibaateiksi, mutta jos tälläistä jäsentä ei ole ollenkaan vakaus on ekvivalentti esteparin $(m, w)$ poissaololle.
Estepari on $m$ ja $w$, jotka eivät ole pari, mutta estävät pariutuksen, jos he olisivat yhdessä paremmassa asemassa kun erillään.
Paras-miehelle solmu on pari, jossa mies saa haluamansa naisen, ja paras-naiselle solmu, jossa nainen saa haluamansa miehen.


\subsection{Vakaitten parien löytäminen}
Esitämme alkuperäisen Gale-Shapley algoritmin.
Miehet aloittavat kosimalla suosikki naistaan. Jokainen nainen joka saa enemmän kuin yhden kosinnan hylkää kaikki paitsi suosikkinsa, miehistä jotka kosi häntä. Nainen ei kuitenkaan hyväksy miestä vielä, vaan pitää häntä "langan päässä" salliakseen mahdollisuuden sille, että parempi mies tulisi vielä kosimaan.

Seuraavassa vaiheessa miehet jotka hylättiin kosivat seuraavia vaihtoehtoja, ja naiset taas hylkäävät kaikki paitsi parhaan vaihtoehdon.

Koska niin kauan kun joku naisista ei ole saanut kosintaa tulee hylkäyksiä ja uusia kosintoja, ja mies ei voi kuin kerran kosia samaa naista, niin lopuksi jokaista naista on kosittu. Kun viimeinen nainen on saanut kosinnan, on kosimisvaihe päättynyt ja jokaisen naisen on hyväksyttävä langan päässä oleva mies. \cite[p. 12-13]{gale62a}

\subsection{Vakaa paritus}
Haluamme todistaa, että aiemman pelin avioliitot ovat vakaita.
Oletetaan, että Matti ja Mari eivät ole naimisissa keskenään, mutta Matti suosii Mariaa hänen oman vaimonsa yli. Matin on siis jossain vaiheessa peliä pitänyt kosia Maria ja Marin jossain vaiheessa hylätä Matti jonkun toisen, paremman parin, edestä. On siis selvää, että Mari suosii omaa aviomiestään Matin sijaan eikä epävakautta synny. \cite[p. 13]{gale62a} Algoritmin aikavaativuus on $O(n^2)$. Täytyy myös huomauttaa, että algoritmi tuottaa tälläisenaan miehille optimaalisia tuloksia, jos halutaan tuottaa naisille optimaalisia tuloksia voidaan osat vaihtaa päittäin algoritmissa ja aloittaa alusta.

\subsection{Vakaitten parien määrä}
Aiemmista tuloksista näemme, että jokaisesta avioliitto pelistä löytyy ainakin yksi vakaa paritus. Tapauksissa joissa pari mieltymykset menevät yhteen, on tasan yksi vakaa paritus. Muissa tapauksissa näitä voi olla maksimissaan eksponentiaalinen määrä. \cite[p. 591]{Balinski}

\subsection{Vakaitten parien vertailu}
Olkoon meillä joukko $M = \{A, B, C\}$ ja $N = \{X, Y, Z\}$. Heidän preferenssit menevät seuraavanlaisesti:\\
$A: Y \leftarrow X \leftarrow Z$, $B: Z \leftarrow Y \leftarrow X$, $C: X \leftarrow Z \leftarrow Y$ \\
$X: B \leftarrow A \leftarrow C$, $Y: C \leftarrow B \leftarrow A$, $Z: A \leftarrow C \leftarrow B$ \\
Meillä on nyt kolme mahdollista vakaata paritusta. $M$ saa ykkösvalintansa ja $N$ kolmannen, $(A, Y), (B, Z), (C, X)$. Molemmat saavat toiseksi parhaan valinnan, $(A, X), (B, Y), (C, Z)$. $N$ saa ykkösvalinnan ja $M$ kolmannen, $(A, Z), (B, X), (C, Y)$. Tässä tapauksessa algoritmi antaa $M$ optimaalisen tuloksen, sillä jokainen joukon $N$ elementti saa tasan yhden kosinnan joka sen täytyy hyväksyä (koska se tulee olemaan paras vaihtoehto mitä se tulee saamaan). Tämä johtuu siitä, että joukon $M$ elementillä on valittavanaan koko joukko $N$, kun taas $N$ joutuu karsimaan kosijoistaan.

\section{Sovellukset}
Vakaa avioliitto algoritmin sovellukset ovat laajat. Sitä voidaan käyttää kaksipuolisissa markkinoissa, kuten vuokra-asunnon hakemisessa, jossa vuokraajat ja hakijat ovat pelaavat joukot.

Toinen esimerkki on elinluvotus tapaukset, jossa suurin tekijä on elimen sopivuus vastaanottajalle. Tässä tapauksessa on siis kaksi joukkoa, luovuttajat ja vastaanottajat, ja haemme vastaanottaja-optimaalista paritusta.


\section{Avioliitto pelien strategia}
Algoritmille annettu alkion preferenssi ei välttämättä vastaa alkion oikeita mieltymyksiä. Algoritmin toispuolisuus johtaa tähän, yksittäinen alkio voi parannella omaa pariutumista antamalla vääriä preferenssejä, kun käytämme "miehet kosii, naiset tarkastelee" -algoritmiä.
Esimerkkinä, neljän miehen ja neljän naisen avioliitto peli, missä naisella $w_{1}$ on seuraavanlainen todellinen mieltymys:
$(m_{1} <_{w1} m_{2} <_{w1} m_{3} <_{w1} m_{4})$, nyt hän voi joutua paritetuksi kenenkä tahansa kansa. Kuitenkin jos hän ilmoittaa seuraavanlaisen mieltymyksen: \\
$(m_{3} <_{w1} m_{4})$ Hän varmistaa parikseen ainakin $m_{3}$:n, selvä parannus aikaisempaan.

Mekanismi on yhden vakaan parituksen valinta. Eikä yhtäkään mekanismia ole olemassa, jossa totuus olisi paras strategia, aikasempi esimerkki riittää todistukseksi.


% --- Back matter ---
%
% bibtex is used to generate the bibliography. The babplain style
% will generate numeric references (e.g. [1]) appropriate for theoretical
% computer science. If you need alphanumeric references (e.g [Tur90]), use
%
 \bibliographystyle{babalpha}
%
% instead.

%\bibliographystyle{babplain}
\bibliography{ref}

\lastpage

\end{document}