% --- Template for thesis / report with tktltiki2 class ---

\documentclass[finnish]{tktltiki2}

% tktltiki2 automatically loads babel, so you can simply
% give the language parameter (e.g. finnish, swedish, english, british) as
% a parameter for the class: \documentclass[finnish]{tktltiki2}.
% The information on title and abstract is generated automatically depending on
% the language, see below if you need to change any of these manually.
% 
% Class options:
% - grading                 -- Print labels for grading information on the front page.
% - disablelastpagecounter  -- Disables the automatic generation of page number information
%                              in the abstract. See also \numberofpagesinformation{} command below.
%
% The class also respects the following options of article class:
%   10pt, 11pt, 12pt, final, draft, oneside, twoside,
%   openright, openany, onecolumn, twocolumn, leqno, fleqn
%
% The default font size is 11pt. The paper size used is A4, other sizes are not supported.
%
% rubber: module pdftex

% --- General packages ---

\usepackage[utf8]{inputenc}
\usepackage{lmodern}
\usepackage{microtype}
\usepackage{amsfonts,amsmath,amssymb,amsthm,booktabs,color,enumitem,graphicx}
\usepackage[pdftex,hidelinks]{hyperref}

% Automatically set the PDF metadata fields
\makeatletter
\AtBeginDocument{\hypersetup{pdftitle = {\@title}, pdfauthor = {\@author}}}
\makeatother

% --- Language-related settings ---
%
% these should be modified according to your language

% babelbib for non-english bibliography using bibtex
\usepackage[fixlanguage]{babelbib}
\selectbiblanguage{finnish}

% add bibliography to the table of contents
\usepackage[nottoc,numbib]{tocbibind}
% tocbibind renames the bibliography, use the following to change it back
\settocbibname{Lähteet}

% --- Theorem environment definitions ---

\newtheorem{lau}{Lause}
\newtheorem{lem}[lau]{Lemma}
\newtheorem{kor}[lau]{Korollaari}

\theoremstyle{definition}
\newtheorem{maar}[lau]{Määritelmä}
\newtheorem{ong}{Ongelma}
\newtheorem{alg}[lau]{Algoritmi}
\newtheorem{esim}[lau]{Esimerkki}

\theoremstyle{remark}
\newtheorem*{huom}{Huomautus}


% --- tktltiki2 options ---
%
% The following commands define the information used to generate title and
% abstract pages. The following entries should be always specified:

\title{Vakaa avioliitto ongelma}
\author{Anis Moubarik}
\date{\today}
\level{Referaatti}
\abstract{}

% The following can be used to specify keywords and classification of the paper:

\keywords{vakaa avioliitto ongelma, vakaat parit}
\classification{} % classification according to ACM Computing Classification System (http://www.acm.org/about/class/)
                  % This is probably mostly relevant for computer scientists

% If the automatic page number counting is not working as desired in your case,
% uncomment the following to manually set the number of pages displayed in the abstract page:
%
% \numberofpagesinformation{16 sivua + 10 sivua liitteissä}
%
% If you are not a computer scientist, you will want to uncomment the following by hand and specify
% your department, faculty and subject by hand:
%
% \faculty{Matemaattis-luonnontieteellinen}
% Outo ääkkösongelma, jos näitä ei määrittele tässä.
 \department{Tietojenkäsittelytieteen laitos} 
 \subject{Tietojenkäsittelytiede}
%
% If you are not from the University of Helsinki, then you will most likely want to set these also:
%
% \university{Helsingin Yliopisto}
% \universitylong{HELSINGIN YLIOPISTO --- HELSINGFORS UNIVERSITET --- UNIVERSITY OF HELSINKI} % displayed on the top of the abstract page
% \city{Helsinki}
%


\begin{document}

% --- Front matter ---

\maketitle        % title page
\makeabstract     % abstract page

\newpage          % clear page after the table of contents


% --- Main matter ---

\section{Mikä on vakaa avioliitto ongelma}

% Write some science here.

Avioliitto ongelmassa on kyse kahdesta erillisestä joukosta, kutsutaan niitä $N$:ksi ja $M$:ksi, naisiki ja miehiksi. Molempien joukkojen jäsenillä on mieltymykset joiden mukaan he haluavat pariutua toisen joukon jäsenien kanssa. Vakaudella näiden joukkojen välisessä parituksessa tarkoitetaan sitä, että alkioille $n \in N$ ja $m \in M$ ei löydy vaihtoehtoista paritusta, jossa $n$ ja $m$ olisivat paremmassa asemassa, kun mitä he ovat pariutettuna keskenään. Artikkelin paino on vakaitten parien analyysissä.

On myös matemaattisesti osoitettu ja todistettu, että vakaa pariutus löytyy aina. Artikkelin tarkoituksena on lähestyä ongelmaa suunnattujen verkkojen kautta.


\section{Vakaat parit}
Jos haluamme esittää ongelman suunnattuna verkkona, meillä on kaksi joukkoa ääreellistä joukkoa, 
$M = \{m_{1}, m_{2},..., m_{|M|}\}$, ja $N = \{n_{1}, n_{2},...,n_{|N|}\}$. Jokaisella joukon jäsenellä on selvät mieltymykset toisen joukon jäsenistä, ja sijoitus parhaasta parista huonoimpaan. Paritus verkossa $\gamma$ on parit $(m, w)$ $m \in M$, $n \in N$, niin että $w$ on $m$:lle sopiva pari ja päinvastoin. Joukkojen jäsenten on mahdollista jäädä selibaateiksi, mutta jos tälläistä jäsentä ei ole ollenkaan vakaus on ekvivalentti esteparin $(m, w)$ poissaololle.
Siis jos $m$ ja $w$, jotka eivät ole pari, estävät pariutuksen, jos he olisivat yhdessä paremmassa asemassa kun erillään.

Esitellään artikkelin \emph{Lemma 1.} Olkoon meillä avioliitto verkko $\gamma$, joka on seuraavanlainen: $m_{1}$ mieltymykset, $w_{1} >_{m_{1}} w_{2} >_{m_{1}} w_{3} >_{m_{1}} w_{4}$.
\\
$m_{2}$ mieltymykset, $w_{4} >_{m_{2}} w_{3} >_{m_{2}} w_{4}$.
\\



\section{Gale-Shapley algoritmi}




% --- Back matter ---
%
% bibtex is used to generate the bibliography. The babplain style
% will generate numeric references (e.g. [1]) appropriate for theoretical
% computer science. If you need alphanumeric references (e.g [Tur90]), use
%
% \bibliographystyle{babalpha}
%
% instead.

\bibliographystyle{babplain}
\bibliography{ref}


\end{document}