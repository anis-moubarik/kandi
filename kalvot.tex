\documentclass{beamer}
\usepackage[utf8]{inputenc}
\title{Vakaa avioliitto -ongelma}
\begin{document}
	\frame{\titlepage}
	\begin{frame}
    \frametitle{Pari ja pariutukset}
    \begin{itemize}
		\item Paria merkitään seuraavasti $(m, n) \in M \times N$ missä $m \in M, n \in N$.
		\item  Pariutus on parien joukko avioliittopelissä.
    \end{itemize}
  \end{frame}
  \begin{frame}
    \frametitle{Vakaat pariutukset}
    \begin{itemize}
    	\item Avioliittopelissä joukkojen $M$ ja $N$ alkiot pistävät vastapuolen alkiot järjestykseen parhaimmasta parista huonoimpaan
		\item Ongelma on löytää vakaa pariutus joukoille $M$ ja $N$.
		\item Pariutus on vakaa, kun meillä ei ole kahta vastakkaisen sukupuolen henkilöä, jotka olisivat mielummin pari keskenään kuin pitäisivät nykyisen parinsa.
		\item Jos meillä on tälläinen pari sitä kutsutaan estepariksi. 
    \end{itemize}
  \end{frame}
  \begin {frame}
  	\frametitle{Esteparin määritelmä}
  	Pariutus $\mu$ on vakaa, jos ei ole olemassa paria $(m_{1}, n_{2})$, jolle
  	\begin{enumerate}
  		\item $n_2 >_{m_{1}} p_{\mu}(m_1)$, ja
  		\item $m_1 >_{n_{2}} p_{\mu}(n_2)$
  	\end{enumerate}
  \end{frame}
  \begin{frame}
    \frametitle{Gale--Shapley-algoritmi}
    \begin{enumerate}
	\item Miehet aloittavat kosimalla mieltymyksiltään parasta naista.
	\item Naiset hylkäävät kaikki paitsi parhaiten sijoitetun miehen.
	\item Nainen ei hyväksy miestä vaan odottaa, jos parempi mies kosisi häntä
	\item Hylätyt miehet kosivat mieltymyksiltään toisiksi parasta naista.
	\item Naiset hylkäävät kaikki paitsi parhaan vaihtoehdon
	\item Niin kauan kuin joku naisista ei ole saanut kosintaa, tulee hylkäyksiä ja uusia kosintoja. Lopuksi jokaista naista on kosittu, koska mies ei voi kuin kerran kosia samaa naista.
	\item Viimeinen nainen on saanut kosinnan ja kosimisvaihe on päättynyt. Jokainen nainen hyväksyy langan päässä olevan kosijan
\end{enumerate}
  \end{frame}
  \begin{frame}
  	\frametitle{Vakaitten pariutuksien joukko}
  \end{frame}
  \begin{frame}
  	\frametitle{Erisuuruiset joukot}
  	\begin{itemize}
  		\item Voimme laajentaa ongelmaa erisuuruisten joukkojen kohdalle.
  		\item Tässä tapauksessa oletamme, että henkilö haluaa parin mielummin kuin olla yksin.
  		\item Olkoon $M$ miesten joukko ja $N$ naisten joukko, ja $|M| = n_x < n_y = |N|$. Pariutus $\mu$ on epävakaa, jos on olemassa mies $m \in X$ ja nainen $n \in Y$ siten että

\begin{enumerate}
	\item $m$ ja $w$ eivät ole pari pariutuksessa $\mu$,
	\item $m$ on joko ilman paria $\mu$:ssä tai suosii naista $w$ enemmän kuin pariaan pariutuksessa $\mu$, ja
	\item $w$ on joko ilman paria $\mu$:ssä tai suosii miestä $m$ enemmän kuin pariaan pariutuksessa $\mu$.
\end{enumerate}
Jokainen vakaa pariutus koostuu $n_x$:stä järjestettyjä pareja, missä $n_x - n_y$ on naimattomien naisten määrä.
  		
  	\end{itemize}
  \end{frame}
  \begin{frame}
  	\frametitle{Pareista kieltäytyminen}
  \end{frame}
  \begin{frame}
  	\frametitle{Sovellukset}
  	\begin{itemize}
  		\item Elinluovutus
  		\item Koulujen valinnat
  		\item Lääketieteen opiskelijat ja opetussairaalat
  	\end{itemize}
  \end{frame}
\end{document}